%%%%%%%%%%%%%%%%%%%%%%%%%%%%%%%%%%%%%%%%%
% Short Sectioned Assignment
% LaTeX Template
% Version 1.0 (5/5/12)
%
% This template has been downloaded from:
% http://www.LaTeXTemplates.com
%
% Original author:
% Frits Wenneker (http://www.howtotex.com)
%
% License:
% CC BY-NC-SA 3.0 (http://creativecommons.org/licenses/by-nc-sa/3.0/)
%
%%%%%%%%%%%%%%%%%%%%%%%%%%%%%%%%%%%%%%%%%

%----------------------------------------------------------------------------------------
%	PACKAGES AND OTHER DOCUMENT CONFIGURATIONS
%----------------------------------------------------------------------------------------

\documentclass[paper=a4, fontsize=11pt]{scrartcl} % A4 paper and 11pt font size

\usepackage[T1]{fontenc} % Use 8-bit encoding that has 256 glyphs
\usepackage{fourier} % Use the Adobe Utopia font for the document - comment this line to return to the LaTeX default
\usepackage[english]{babel} % English language/hyphenation
\usepackage{amsmath,amsfonts,amsthm} % Math packages
\usepackage[pdftex]{graphicx}
\graphicspath{{./}}

\usepackage{lipsum} % Used for inserting dummy 'Lorem ipsum' text into the template

\usepackage{sectsty} % Allows customizing section commands
\allsectionsfont{\centering \normalfont\scshape} % Make all sections centered, the default font and small caps

\usepackage{fancyhdr} % Custom headers and footers
\pagestyle{fancyplain} % Makes all pages in the document conform to the custom headers and footers
\fancyhead{} % No page header - if you want one, create it in the same way as the footers below
\fancyfoot[L]{} % Empty left footer
\fancyfoot[C]{} % Empty center footer
\fancyfoot[R]{\thepage} % Page numbering for right footer
\renewcommand{\headrulewidth}{0pt} % Remove header underlines
\renewcommand{\footrulewidth}{0pt} % Remove footer underlines
\setlength{\headheight}{13.6pt} % Customize the height of the header

\numberwithin{equation}{section} % Number equations within sections (i.e. 1.1, 1.2, 2.1, 2.2 instead of 1, 2, 3, 4)
\numberwithin{figure}{section} % Number figures within sections (i.e. 1.1, 1.2, 2.1, 2.2 instead of 1, 2, 3, 4)
\numberwithin{table}{section} % Number tables within sections (i.e. 1.1, 1.2, 2.1, 2.2 instead of 1, 2, 3, 4)

%\setlength\parindent{0pt} % Removes all indentation from paragraphs - comment this line for an assignment with lots of text

%----------------------------------------------------------------------------------------
%	TITLE SECTION
%----------------------------------------------------------------------------------------

\newcommand{\horrule}[1]{\rule{\linewidth}{#1}} % Create horizontal rule command with 1 argument of height

\title{	
\normalfont \normalsize 
\textsc{Network Virtualization and Data Center Networks} \\ [25pt] % Your university, school and/or department name(s)
\horrule{0.5pt} \\[0.4cm] % Thin top horizontal rule
\huge Assignment 3: Software Defined Networking\\ % The assignment title
\horrule{2pt} \\[0.5cm] % Thick bottom horizontal rule
}

\author{Erik Henriksson, Christoph Burkhalter} % Your name

\date{\normalsize\today} % Today's date or a custom date

\begin{document}

\maketitle % Print the title

\section{Forwarding in a Single Switch}

In a single switch we use the hostlist file to set up flows and since the routing is trivial we don't need a fancy routing algorithm. We use the match object to route the packets by IP to the correct port specified in the hostlist file. 

\section{Forwarding in a Binary Tree}

The controller for the binary tree case is almost identical to the Fat-Tree-Like controller, except that links are not chosen randomly because there is only one possible link.

\section{Forwarding along the Shortest Path in a Fat-Tree-Like Topology}

The implementation for the controller that deals with Fat-Tree-Like topologies has an initialization phase and several function which are called based on different events. The initialization loads the hostlist.csv file and registers to the following events: ConnectionUp, PacketIn and LinkEvent.

ConnectionUp is called whenever a new switch joins the network and the new connection is added to the list. PacketIn could be used to deal with packets that aren't delivered, however, in the current implementation these packets are dropped. The last event is the LinkEvent, this indicates that there is a new link between switches. The controller removes all existing flows from the switches and calculates the flow entries based on the new network topology. 

The following algorithm will set the flow entries for all switches. It is based on the idea that for each host destination ip, every switch needs to now where to forward a packet for that host.
So for each host the algorithm performs these steps:
First the list of switches is copied. During the algorithm, a switch must be in one of the following states: not visited, last visited or finished. At the beginning all switches are marked as not visited except for the switch that is connected to the host. This information is stored in the hostlist.csv file. So for the first switch the flow entry is set to forward all packages with the host ip address as destination address to the host. Additionally, the state of this switch is set to last visited.

Next, based on the adjacency list, each link from a switch which is not visited to a switch which is in the last visited state is added to the list with new links. Then the switches with state last visited are set to finished. Each switch that has a link in the generated list will get a flow entry to forward packets with the destination ip address of the host to the switch that was in the last visited state. Last, these switches are set the last visited. This is repeated until there are no new links. If there is more than one link for a single switch, the link is chosen randomly to balance the system.

With this algorithm, the shortest path based on number of hops is chosen for every possible connection.



\end{document}
